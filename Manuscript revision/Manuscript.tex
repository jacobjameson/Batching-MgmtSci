%%%%%%%%%%%%%%%%%%%%%%%%%%%%%%%%%%%%%%%%%%%%%%%%%%%%%%%%%%%%%%%%%%%%%%%%%%%%
%% Author template for INFORMS Journal on Data Science (ijds) [interim solution; new styles under construction]
%% Mirko Janc, Ph.D., INFORMS, mirko.janc@informs.org
%% ver. 0.91, March 2015 - updated November 2020 by Matthew Walls, matthew.walls@informs.org
%% Adapted for rticles by Rob J Hyndman Rob.Hyndman@monash.edu. Dec 2021
%%%%%%%%%%%%%%%%%%%%%%%%%%%%%%%%%%%%%%%%%%%%%%%%%%%%%%%%%%%%%%%%%%%%%%%%%%%%
\documentclass[,,nonblindrev]{informs}

\OneAndAHalfSpacedXI
%%\OneAndAHalfSpacedXII % Current default line spacing
%%\DoubleSpacedXII
%%\DoubleSpacedXI

%% BEGIN MY ADDITIONS %%
\usepackage{hyperref}

% tightlist command for lists without linebreak
\providecommand{\tightlist}{%
  \setlength{\itemsep}{0pt}\setlength{\parskip}{0pt}}



\usepackage{booktabs}
\usepackage{tabularx}
\usepackage{graphicx}
\usepackage{makecell}
\usepackage{float}
\usepackage{tikz}
\usepackage{siunitx}
\usepackage{tablefootnote}
\usepackage{longtable}
\usepackage{threeparttable}
\usepackage{natbib}
\usepackage{caption}
\usepackage{adjustbox}
\usepackage{multirow}
\usepackage{float}
\usepackage{placeins}
\usepackage[]{mdframed}

%% END MY ADDITIONS %%


% Natbib setup for author-year style
\usepackage{natbib}
 \bibpunct[, ]{(}{)}{,}{a}{}{,}%
 \def\bibfont{\small}%
 \def\bibsep{\smallskipamount}%
 \def\bibhang{24pt}%
 \def\newblock{\ }%
 \def\BIBand{and}%


%% Setup of theorem styles. Outcomment only one.
%% Preferred default is the first option.
\TheoremsNumberedThrough     % Preferred (Theorem 1, Lemma 1, Theorem 2)
%\TheoremsNumberedByChapter  % (Theorem 1.1, Lema 1.1, Theorem 1.2)
\ECRepeatTheorems

%% Setup of the equation numbering system. Outcomment only one.
%% Preferred default is the first option.
\EquationsNumberedThrough    % Default: (1), (2), ...
%\EquationsNumberedBySection % (1.1), (1.2), ...

% For new submissions, leave this number blank.
% For revisions, input the manuscript number assigned by the on-line
% system along with a suffix ".Rx" where x is the revision number.
\MANUSCRIPTNO{}

%%%%%%%%%%%%%%%%
\begin{document}
%%%%%%%%%%%%%%%%

% Outcomment only when entries are known. Otherwise leave as is and
%   default values will be used.
%\setcounter{page}{1}
%\VOLUME{00}%
%\NO{0}%
%\MONTH{Xxxxx}% (month or a similar seasonal id)
%\YEAR{0000}% e.g., 2005
%\FIRSTPAGE{000}%
%\LASTPAGE{000}%
%\SHORTYEAR{00}% shortened year (two-digit)
%\ISSUE{0000} %
%\LONGFIRSTPAGE{0001} %
%\DOI{10.1287/xxxx.0000.0000}%

% Author's names for the running heads
% Sample depending on the number of authors;
\RUNAUTHOR{%
Jameson, Saghafian, Huckman, Hodgson
 and Baugh
}
% \RUNAUTHOR{Jones and Wilson}
% \RUNAUTHOR{Jones, Miller, and Wilson}
% \RUNAUTHOR{Jones et al.} % for four or more authors
% Enter authors following the given pattern:
%\RUNAUTHOR{}

\RUNTITLE{Image Batching in EDs}

\TITLE{The Impact of Batching Advanced Imaging Tests in Emergency
Departments}

\ARTICLEAUTHORS{%
\AUTHOR{Jacob Jameson}
\AFF{Harvard Kennedy School, Harvard University, \EMAIL{}}

\AUTHOR{Soroush Saghafian}
\AFF{Harvard Kennedy School, Harvard University, \EMAIL{}}

\AUTHOR{Robert Huckman}
\AFF{Harvard Business School, Harvard University, \EMAIL{}}

\AUTHOR{Nicole Hodgson}
\AFF{Department of Emergency Medicine, Mayo Clinic of Arizona, \EMAIL{}}

\AUTHOR{Joshua Baugh}
\AFF{Department of Emergency Medicine, Massachusetts General
Hospital, \EMAIL{}}

%
}

\ABSTRACT{Using detailed electronic health record data from two major
U.S. emergency departments (EDs), we use practice variation across
physicians to uncover the operational impact of discretionary batch
ordering of imaging tests. We find that quasi-random assignment of a
patient to an ED physician who is a ``batcher'' (top decile) versus a
``sequencer'' (bottom decile) causally increases the patient's length of
stay, time to disposition, and the number of imaging tests endured.
Instrumental variable results show that discretionary batching increases
length of stay by approximately 65\%, increases imaging tests by 88\%,
and has no effect on 72-hour returns. We find evidence that the impact
of batching on length of stay is heavily mediated by the number of
additional imaging tests performed and the probability of hospital
admission after the ED service. Results suggest this ordering strategy
may lead to clinical decision-making that introduces bottlenecks in
patient flow. Conversely, standard practice, which preserves diagnostic
flexibility by allowing information from initial tests to guide
subsequent decisions, offers an ``information gain'' advantage over
batch ordering: the information obtained from a prior test allows the
elimination of the need to order some future tests. Put together, our
findings indicate that discretionary batch ordering is not an efficient
strategy for managing diagnostic imaging in emergency care, and
interventions to reduce unnecessary batching may improve both
operational performance and resource utilization. blind: false}

\KEYWORDS{Emergency Department operations; Diagnostic imaging; Batch
ordering; Physician practice patterns; Patient outcomes; Health care
efficiency}

\maketitle


\section{Introduction}\label{sec:1}

Advanced imaging is simultaneously an emergency department's (ED) most
powerful diagnostic tool and one of its most significant operational
bottlenecks \citep{Rogg2017}. Advanced diagnostic imaging has risen
dramatically over the past two decades, transforming from a limited
resource to a cornerstone of emergency care
\citep[\citet{smith-bindman2019trends}]{Juliusson2019}. However, this
transformation has intensified operational challenges in EDs, where the
complexity stems from multiple modifiable and non-modifiable
constraints: limited equipment availability, complex scheduling
requirements across different modalities, extended wait times for both
image acquisition and interpretation, and growing backlogs in
radiologist reading queues. Because of this, diagnostic imaging
represents one of the most resource-intensive and operationally complex
components of ED care \citep[\citet{baloescu2018diagnostic},
\citet{poyiadji2023diagnostic}]{mills2015optimizing}.

Diagnostic imaging is not without consequences for patients. Imaging
bottlenecks can significantly impact ED length of stay (LOS)
\citep{Cournane2016}, and undergoing advanced imaging may expose
patients to higher costs, increased radiation exposure, increased
incidental findings that may lead to unnecessary follow-up testing,
contrast-induced nephropathy, and contrast-induced allergic reactions
\citep[\citet{Raja2014}]{valtchinov2019use}. Given these risks and
resource constraints, efficient diagnostic testing management is
critical for patient outcomes and ED operations \citep{naseim2015}.
Physicians have considerable discretion over how they order these tests,
and wide variation in diagnostic test ordering behavior has been well
documented \citep[\citet{Solomon1998}, \citet{Wennberg1984},
\citet{Daniels1977}]{Miller1994}. However, little is known about how to
manage test ordering strategies when this discretion exists.

We consider an ED physician's decision to order imaging tests for their
patient as an optimization problem, where the physician must balance the
tradeoffs between the advantages of ordering multiple tests
simultaneously (batch) at the start of the patient encounter (e.g.,
expediting the diagnostic process if the tests are eventually necessary
for diagnosis and disposition) and its disadvantages (e.g., increasing
the total time spent in the ED because of unnecessary tests)
\citep[\citet{Perotte2018}, \citet{Lyu2017},
\citet{Traub2018}]{Tamburrano2020}.

Batch ordering stands in contrast to the more standard practice of
physicians sequentially ordering one test at a time, reviewing the
results, and then deciding whether to order additional tests based on
the information obtained from the previous test. While this standard
practice may serve as a natural filter to prevent unnecessary testing,
it can also result in longer delays than batching when multiple tests
are ultimately needed. This tension between potential efficiency gains
and the risk of unnecessary testing is significant, given growing
concerns about imaging overutilization in emergency care
\citep[\citet{mills2015optimizing}]{baloescu2018diagnostic}. Beyond
avoiding delays, the decision to batch should also be made with
consideration of immediate operational implications and broader
quality-of-care considerations \citep{Feizi2023}. This decision is
inherently complex, requiring physicians to weigh time-sensitive
diagnostic needs against resource constraints and the risk of ordering
tests that may prove unnecessary once earlier results become available.

This paper explores the causal effects of batch ordering advanced
imaging tests on operational performance and patient outcomes in the ED.
Specifically, we focus on batch orders that include multiple types of
imaging tests that cannot be run in a single scanning session. Using
data from two leading U.S. hospitals---Mayo Clinic and Massachusetts
General Hospital---we first provide evidence of significant variation in
emergency physicians' batching behavior. Batching varies significantly
among physicians working in the same ED, treating patients with the same
complaint and severity (Figure \ref{fig: Variation}). By making use of
this variation, we next investigate whether being seen by a ``batcher''
or a ``sequencer'' physician has implications in terms of performance
metrics such as length of stay (LOS), number of imaging studies, 72-hr
rate of return, and disposition decision (being admitted to the hospital
or discharged home post ED service). Finally, we shed light on
circumstances in which physicians are more likely to batch-order imaging
tests.

\begin{figure}[t!] 
\centering
\caption{Physician Variation in Batch Ordering Imaging Tests}  
\includegraphics[width=\textwidth]{../outputs/figures/fig1_boxplot.png}
\label{fig:Variation}
\begin{flushleft}
\footnotesize{\textit{Notes:} This figure highlights the marked differences among Mayo Clinic ED physicians in their propensity to batch order imaging tests. Batch rates are crude rates calculated by dividing the number of patient encounters in which the physician batch-ordered imaging tests for a complaint by the number of encounters with that complaint.}
\end{flushleft}
\end{figure}

\subsection{Challenges and Empirical
Strategy}\label{challenges-and-empirical-strategy}

While the importance of understanding batching behavior in EDs is
evident, generating causal evidence of its effects on patient outcomes
presents several empirical challenges. First, physicians' decisions to
batch order tests are endogenous. For example, the choice to batch may
be correlated with unobservable patient characteristics, physician
workload, or ED conditions that independently affect outcomes. Second,
studying the timing and sequencing of diagnostic testing requires
granular operational data that captures precise timestamps of test
orders, completions, and clinical decisions, many of which are not
tracked in traditional claims databases. Third, while prior studies have
leveraged quasi-random patient assignment in EDs
\citep[\citet{Gowrisankaran2022},
\citet{Coussens2024}]{eichmeyer2022pathways}, establishing true
quasi-randomization requires a detailed understanding of institutional
assignment mechanisms that may vary across settings. Thus, providing
causal claims that can be generalized beyond a single ED is not
straightforward.

Our study addresses these challenges through several unique features.
First, we utilize detailed electronic health record data from two
leading U.S. hospitals (Mayo Clinic and Massachusetts General Hospital
(MGH)) that capture the complete temporal sequence of clinical
decisions, including exact timestamps of test orders, results
availability, and disposition decisions (Table \ref{tab:desc_stats}).
This granularity allows us to precisely measure batch ordering behavior
and its downstream effects on patient flow. Second, our primary analysis
leverages Mayo Clinic's rotational patient assignment system, which
verifies that patients are randomly assigned to physicians via a
round-robin algorithm that does not consider patient characteristics or
physician workload \citep[\citet{Traub2016}]{traub2016emergency}. We use
this random assignment at the Mayo Clinic to gain deeper insights into
the causal impacts of being seen by a ``batcher.'' We then validate our
findings using data from MGH, which employs a wholly different and
non-random patient-to-physician assignment. We do so by creating
suitable controls that enable us to construct a quasi-random assignment
mechanism and to investigate the validity of our main findings across
two different study sites.

\begin{table}[t!]
\centering
\caption{Descriptive Statistics of Emergency Department Encounters}
\label{tab:desc_stats}
\begin{threeparttable}
\begin{tabular}{p{9cm}cc}
\toprule
Variable & Mayo Clinic & MGH \\
 & (Median [IQR]) & (Median [IQR]) \\
Patients encounters$^a$ & n = 48,854 & n = 111,710 \\
\midrule
\textit{Panel A. Patient Severity} & & \\
Tachycardic & 19.2\% & 21.3\% \\
Tachypneic & 8.8\% & 5.4\% \\
Febrile & 2.2\% & 1.5\% \\
Hypotensive & 1.4\% & 1.0\% \\
Emergency Severity Index & 2.8 [2, 3] & 2.8 [2, 3] \\
\\
\textit{Panel B. Patient Demographics} & & \\
Male & 46.5\% & 51.3\% \\
Race: White & 88.4\% & 61.3\% \\
Race: Black & 4.2\% & 12.3\% \\
Race: Asian & 3.05\% & 4.89\% \\
Arrival age & 57.7 [43, 74] & 49.6 [32, 65] \\
\\
\textit{Panel C. Diagnostic Tests and Outcomes} & & \\
X-ray performed & 43.3\% & 40.1\% \\
Ultrasound performed & 11.3\% & 18.3\% \\
Non-contrast CT performed & 35.5\% & 30.5\% \\
Contrast CT performed & 17.7\% & 13.0\% \\
MRI performed$^{b}$ & --- & 6.3\% \\
Labs ordered & 73.7\% & 84.9\% \\
Time from arrival to triage (mins) & 8.0 [4, 10] & 12.2 [3, 13] \\
LOS (min) & 246 [152, 306] & 426 [241, 1006] \\
Time to disposition (min) & 185 [126, 257] & --- \\
Treatment Time (min) & 160 [105, 227] & --- \\
Patient discharged & 66.8\% & 65.1\% \\
Patient admitted & 18.7\% & 22.8\% \\
Patients revisited within 72 hours & 3.8\% & 3.1\% \\
Order to result$^{c}$: X-Ray (mins) & 67.2 [36, 79] & 63.4 [31.4, 111.7] \\
Order to result: Ultrasound (mins) & 165 [71, 150] & 119 [71, 213] \\
Order to result: Contrast CT (mins) & 142 [86, 153] & 167.3 [112.0, 260.6] \\
Order to result: Non-Contrast CT (mins) & 89.7 [50, 102] & 185.3 [116.0, 290.2] \\
Order to result: MRI (mins) & --- & 374.7 [229.5, 683.7] \\
\bottomrule
\end{tabular}
\begin{flushleft}
\footnotesize \textit{Notes:} This table reports summary statistics for emergency department visits during the study period. Values are presented as mean (IQR) when available. Vital signs were categorized as follows: tachycardia (pulse more significant than $100$), tachypnea (respiratory rate greater than $20$), fever (temperature greater than $38^\circ C$), and hypotension (systolic blood pressure less than $90$).
\footnotesize $^a$The number of patient encounters is calculated as the number of unique patient visits during the study period.
\footnotesize $^b$MRI data is only available for MGH.
\footnotesize $^c$Order to result times are calculated as the difference between the time the test was ordered and the time the result was available in the electronic health record. This does not account for the time it takes for the radiologist to review the results.
\end{flushleft}
\end{threeparttable}
\end{table}

Our empirical strategy closely follows the literature that relies on the
quasi-random assignment of agents to cases, often referred to as the
``judges design.'' \citep[\citet{dobbie2018effects}]{Dahl2014}. Studies
in this literature typically exploit variation in judges' sentencing
leniency within the same court. Similarly, we exploit variation in
batching across physicians working in the same ED through a measure we
define as ``batch tendency.'' In its reduced form, under the assumption
of random or quasi-random assignment, this approach allows us to
identify the causal effect of being assigned to different types of
physicians (i.e., batcher or sequencer). Under additional assumptions,
an instrumental variable (as we will define it) allows us to estimate
the causal effect of the decision to batch on various measures of ED
performance.

Our instrumental variables approach provides a local average treatment
effect (LATE) for batch ordering, the effect of batch ordering on the
subset of patients whose batching decision depends on which physician
they are randomly assigned to. This is particularly important in the
context of the ED, where there are clear cases where batch ordering is
necessary (e.g., when multiple tests are essential and almost every
physician would batch) and cases where it is not (e.g., when one or no
tests are needed and virtually no physician would batch). By estimating
the effect for marginal patients whose batching decisions depend on the
provider, we gain deeper insights into the critical implications of
batch ordering driven by physician discretion.

\subsection{Main Findings and
Contributions}\label{main-findings-and-contributions}

Whether batch ordering improves or hinders ED efficiency is an active
debate among emergency physicians. Proponents argue that initiating
multiple diagnostic processes simultaneously reduces total processing
time when multiple tests are ultimately needed; skeptics contend that
this approach foregoes the information value of sequential results and
leads to unnecessary testing. Our results provide causal evidence that,
at least for discretionary decisions at the margin of physician
judgment, batch ordering negatively impacts operational metrics. In
particular, our results show that the marginal batched patient
experiences a 65\% increase in total ED LOS and a 69\% increase in time
to disposition compared to patients managed through standard practice.
Furthermore, discretionary batch ordering leads to 1.2 additional
imaging tests per patient encounter (an 88\% increase) and a 40
percentage point increase in admission probability. These effects remain
robust even after adjusting for patient and physician characteristics,
laboratory testing intensity, and ED capacity. Through mediation
analysis, we find that increased testing volume is a key driver of the
higher LOS caused by discretionary batch ordering. We find evidence that
this effect is further mediated by the higher likelihood of admission,
which can increase LOS through boarding time.

Examining the drivers of batch ordering, we find that physicians are
more likely to batch-order tests earlier in their shifts and for
patients with higher acuity and more complex chief complaints. This
batching tendency, however, varies systematically across physicians and
persists even after controlling for patient mix, ED conditions, and
observable physician characteristics. Finally, we observe that batching
rates decline modestly during periods of major overcapacity (12.4\%
vs.~14.8\% during normal operations), suggesting physicians may become
more selective in their batching decisions under resource constraints.

Our results indicate that, despite the perceived workflow advantages of
initiating multiple diagnostic processes simultaneously, batch ordering
results in significantly longer processing times and increased resource
utilization without corresponding improvements in patient outcomes. We
validate these findings through heterogeneity analyses across seven
high-batching chief complaint categories, finding consistent patterns
--- increased imaging without efficiency gains or quality improvements
--- across all clinical scenarios. These findings have important
implications, highlighting that ED managers should consider strategies
to reduce discretionary batching to improve diagnostic testing
workflows. We shed light on these strategies and provide actionable
insights on how ED managers can achieve them.

\section{Related Literature}\label{related-literature}

In EDs, physicians face a fundamental choice in how they sequence
diagnostic imaging tests: they can order multiple tests simultaneously
(``batch ordering'') or order them sequentially as information
accumulates. Batch ordering occurs when a physician orders a
comprehensive set of diagnostic tests at the start of a patient
encounter, typically covering a broad range of potential diagnoses. This
contrasts with standard practice, where tests are ordered one at a time,
with each subsequent test decision informed by prior test results.

\subsection{Physician Decision-Making Under
Constraints:}\label{physician-decision-making-under-constraints}

ED physicians operate under significant time pressure and heavy
workload, which substantially influences their decision-making. Workload
management and carefully balancing the tradeoffs between speed and
quality of care are of high importance for physicians making decisions
in complex environments, such as the ED
\citep[\citet{leppink2019mental}]{Saghafian2018}, and these decisions
lead to multiple documented effects on physician performance. For
instance, physicians may resort to ordering additional diagnostic tests
when faced with limited patient interaction time, a practice that
requires less immediate critical thinking than direct clinical
assessment \citep[\citet{pines2009trends}]{batt2016early}.

Moreover, increased stress and frequent interruptions can disrupt
systematic decision-making and impair complex diagnostic reasoning
\citep[\citet{bendoly2011linking}]{chisholm2000emergency}. Frequent task
switching to manage multiple simultaneous demands can exacerbate
cognitive strain and decision fatigue, prompting physicians to batch
diagnostic tests and defer intricate diagnostic decisions until
comprehensive results are available \citep[\citet{skaugset2016can},
\citet{jaeker2020value}]{kc2013does}. By ordering multiple tests
upfront, physicians can defer complex diagnostic reasoning until all
results are available, potentially reducing the cognitive strain of
repeated task-switching and decision-making under uncertainty. A
cognitive load theory of batching suggests that batching is more
prevalent during periods of high workload or complexity.

Significant variation in ED physician testing and admitting practices
has also been documented \citep[\citet{Coussens2024},
\citet{Smulowitz2021}]{hodgson2018are}. This variation also extends to
batch ordering, where physicians differ systematically in their
propensity to order multiple imaging tests simultaneously
\citep{jameson2024variation}. However, understanding the drivers and
operational impact of this variation remains a significant gap in the
literature. Our study aims to fill this gap by examining the factors
that drive batch ordering behavior and exploiting physician variation to
identify the causal effects of batch ordering on ED operations and
patient outcomes.

\subsection{Discretionary Behavior and Task Scheduling in Healthcare
Operations}\label{discretionary-behavior-and-task-scheduling-in-healthcare-operations}

Recent work has examined how operational factors influence physicians'
discretionary behavior in test ordering, revealing that decisions about
diagnostic intensity are shaped by multiple operational pressures
\citep{soltani2022does}. Studies have found that test utilization varies
with peer observation \citep{song2017closing}, workload \citep{Deo2019},
and the presence of justification requirements \citep{jaeker2020value}.
While additional tests can improve diagnostic accuracy, they also extend
ED LOS and potentially exacerbate congestion \citep{chan2018efficiency}.
This tension is particularly acute in imaging decisions, where test
sequencing can significantly impact patient flow and resource
utilization \citep{Cournane2016}.

The decision to batch order tests represents a specific form of
discretionary task ordering that has received limited attention in
healthcare operations. While prior work has examined discretionary task
ordering in other contexts \citep[\citet{Ibanez2020}]{Ibanez2018}, the
unique constraints of ED imaging---such as capacity limitations, varying
processing times across modalities, and the inability to run different
imaging types on a patient simultaneously---make these decisions
particularly consequential. A growing literature examines how workers
exercise discretion over task ordering to improve system performance
\citep[\citet{Campbell2011}]{vanDonselaar2010ordering} but also how such
discretion can sometimes lead workers to ``choose the wrong task
operationally'' \citep{boudreau2003interface}.

The implications of batch ordering also connect to broader theoretical
work on task scheduling in resource-constrained environments. While
batching strategies often reduce setup times and improve throughput in
manufacturing settings \citep{Fowler2022}, applying these principles to
healthcare operations introduces unique complexities. While batching may
streamline the diagnostic process by initiating multiple diagnostic
processes simultaneously \citep{song2017closing}, recent evidence
suggests it may lead to increased testing volumes that could overwhelm
imaging departments and extend wait times
\citep[\citet{saghafian2015operations}]{Jessome2020}. The information
value of sequential testing ---where results from initial tests inform
the need for subsequent ones ---creates a fundamental tension between
operational and diagnostic efficiency that has not been well studied.
Our analysis provides novel evidence on this tradeoff, showing how
different test ordering strategies affect operational metrics and
clinical decision-making. Our study advances this literature by
providing causal evidence on how physicians' test sequencing decisions
affect ED performance. Furthermore, we identify specific mechanisms
through which batch ordering affects operational performance, allowing
us to distinguish between efficiency gains from parallel processing and
potential losses from increased diagnostic intensity.

\subsection{Hypothesis Development}\label{hypothesis-development}

Building on the literature reviewed above, we develop formal hypotheses
about how batch ordering affects ED operations. While prior work has
identified the mechanisms driving batching behavior and its potential
consequences, the net effects remain an open empirical question. Our
theoretical framework centers on the fundamental tradeoff between the
perceived efficiency of parallel processing and the information value of
sequential testing.

\subsubsection{Information Value and Test
Volume}\label{information-value-and-test-volume}

The decision to batch or sequence tests fundamentally involves whether
to preserve the option value of information. Sequential testing allows
each test result to inform subsequent decisions, potentially eliminating
unnecessary tests. When physicians batch tests upfront, they commit to a
diagnostic pathway before information unfolds, thereby forfeiting this
option value.

Physicians under high workload face cognitive strain from task switching
and may batch tests to defer complex diagnostic reasoning
\citep[\citet{skaugset2016can}]{kc2013does}. However, this cognitive
convenience comes at a cost. Without the filtering mechanism of
sequential information revelation, physicians must rely solely on their
initial assessment. \citep{lam2020why} identify this as a key driver of
overtesting---when facing diagnostic uncertainty, physicians order
comprehensive test batteries rather than allowing initial results to
guide subsequent testing. Given the documented variation in physician
testing intensity \citep{hodgson2018are}, with some physicians ordering
twice as many tests as their peers, batching likely amplifies these
tendencies by removing the natural stopping points that sequential
results provide. Therefore:

\begin{quote}
\small
\textit{\textbf{Hypothesis 1.} Batch ordering will increase the total number of imaging tests performed compared to standard practice due to the loss of information value from initial test results.}
\end{quote}

\subsubsection{Processing Time and Operational
Flow}\label{processing-time-and-operational-flow}

While batching strategies reduce setup times in manufacturing
\citep{Fowler2022}, the ED imaging context presents unique operational
constraints as noted in our review. Different imaging modalities require
separate equipment and cannot be performed simultaneously
\citep{Jessome2020}. This creates a fundamental bottleneck: batched
orders must still be executed sequentially, but with a larger committed
workload that cannot be adjusted based on emerging information.

Moreover, the cognitive load literature suggests that processing
multiple test results simultaneously increases decision complexity
\citep{kc2013does}. When physicians receive multiple results at once
rather than sequentially, they must integrate more information
simultaneously, potentially lengthening the diagnostic reasoning
process. This ``information overload'' effect, combined with the
additional tests ordered as predicted in H1, suggests that batching may
paradoxically increase rather than decrease processing times:

\begin{quote}
\small
\textit{\textbf{Hypothesis 2.} Batch ordering will increase patient length of stay and time to disposition compared to standard practice, as the operational constraints of imaging and increased test volume outweigh any potential benefits of parallel processing.}
\end{quote}

\subsubsection{Clinical Decision-Making and
Disposition}\label{clinical-decision-making-and-disposition}

The medical literature documents ``diagnostic momentum''---where
abnormal findings, even if clinically insignificant, drive further
workup and more conservative clinical decisions
\citep[\citet{featherston2020decision}]{coen2022clinical}. When
physicians batch order and receive multiple results simultaneously, they
encounter more opportunities for incidental findings that may influence
disposition decisions
\citep[\citet{berlin2011incidentaloma}]{lumbreras2010incidental}. As our
review noted, physicians facing uncertainty and potential legal
consequences may opt for more conservative disposition decisions
\citep[\citet{lam2020why}]{rao2012overuse}. The simultaneous arrival of
multiple test results, particularly with incidental findings, may
trigger defensive medicine behaviors:

\begin{quote}
\small
\textit{\textbf{Hypothesis 3.} Batch ordering will increase hospital admission rates through increased diagnostic intensity and the influence of incidental findings on clinical decision-making.}
\end{quote}

\subsubsection{Contextual Moderators}\label{contextual-moderators}

The literature on physician behavior under capacity constraints
consistently shows that resource scarcity forces more selective
decision-making \citep[\citet{kc2009impact}]{kuntz2014stress}. When EDs
face severe overcrowding, the operational pressures documented in our
review intensify. Under these conditions, physicians may reserve
batching for cases where it is clinically essential rather than
convenient:

\begin{quote}
\small
\textit{\textbf{Hypothesis 4.} The effects of batch ordering on LOS and test volume will be attenuated under conditions of major ED overcapacity, as physicians become more selective in their batching decisions.}
\end{quote}

These hypotheses provide testable predictions that we examine using our
quasi-experimental design. By leveraging variation in physicians'
batching tendencies under random patient assignment, we can determine
whether these theoretical mechanisms manifest in actual ED operations.

\section{Setting, Data, and Models}\label{sec:3}

\subsection{Empirical Setting}\label{sec:3.1}

Our study uses data from two large U.S. emergency departments (EDs): the
Mayo Clinic of Arizona and Massachusetts General Hospital (MGH). The MGH
dataset, which includes 129,489 patient encounters from November 10,
2021, through December 10, 2022, provides a robust sample for validating
the generalizability of our findings. However, our primary analysis
focuses on the Mayo Clinic data due to its unique random
patient-physician assignment, which enables stronger causal inference.
More specifically, the Mayo Clinic ED employs a sophisticated
computerized rotational patient assignment algorithm that addresses many
of the empirical challenges previously identified
\citep{Traub2016, traub2016emergency, Traub2018}. The system
automatically assigns patients to physicians 60 seconds after
registration through the electronic health record system, following a
strict rotational protocol. At shift start, each physician receives four
consecutive patients to establish an initial patient load, after which
they enter rotation with other on-duty physicians. Critically, these
assignments are based solely on arrival time---the algorithm does not
consider patient demographics, chief complaint, Emergency Severity Index
score, physician workload, or the acuity of patients who have been
recently assigned. To maintain system integrity, physicians receive no
new patients during their final 120 minutes and are capped at 18
patients per shift. The rotation order follows a predetermined schedule
that varies across shifts to ensure fairness over time.

This rotational mechanism achieves the quasi-randomization necessary for
causal inference. Unlike settings where patient-physician matching may
be influenced by triage decisions, physician preferences, or informal
routing practices, the Mayo Clinic's algorithmic assignment removes
discretion from the matching process. We establish that, conditional on
arrival time, patient-physician matching is effectively random---a
critical requirement for our identification strategy that distinguishes
our study from observational analyses where endogenous matching could
confound the effects of physician discretion.

The data include information on the timing of test orders, test results,
patient disposition, and other important triage metrics and demographic
features. We focus on imaging tests (x-rays, contrast CT scans,
non-contrast CT scans, and ultrasound) because, unlike laboratory tests,
these tests cannot be run simultaneously on a given patient due to the
different equipment and settings required. Therefore, the operational
implications of batch ordering imaging tests are more pronounced. We
exclude MRI from our Mayo analysis because an institutional policy
requires either inpatient admission for urgent MRIs or outpatient
ordering for non-urgent cases, resulting in negligible ED MRI use. MGH
does not have this policy, so we included MRI in their generalizability
analysis.

\subsection{Data}\label{data}

Our primary data comes from the Mayo Clinic of Arizona ED, a tertiary
care hospital without obstetrical services, an inpatient pediatrics
unit, or a trauma designation. During the study period (October 6, 2018,
through December 31, 2019), the ED recorded 48,854 visits, managed
across 26 treatment rooms and up to 9 hallway spaces. The department is
staffed exclusively by board-eligible or board-certified emergency
physicians (EPs), a rare yet ideal setting for our study. Many EDs are
staffed by a mix of EPs and non-EPs (e.g., Nurse Practitioners and
Physician Assistants), both of whom are responsible for ordering tests,
which may introduce confounding factors. The Mayo Clinic ED is unique in
that only EPs can order tests, eliminating the potential for confounding
by provider type. Furthermore, as mentioned earlier, the ED uses a
randomized patient-to-EP assignment, eliminating some other potential
sources of confounding.

We conducted a retrospective review of the comprehensive ED operations
data, coinciding with the initiation of a new electronic medical record.
The data includes detailed patient demographics, chief complaints, vital
signs, emergency severity index (ESI), LOS, timestamps, and resource
utilization metrics. This period was chosen to provide a robust data set
while excluding the influence of the coronavirus pandemic. The data is
summarized in Table \ref{tab:desc_stats}. Hourly patient arrival rates
to the ED are shown in Appendix Figure A1. LOS is measured from arrival
to departure from the ED. During periods when inpatient beds are filled
(i.e., when patients requiring inpatient admission must ``board'' in the
ED due to lack of inpatient beds), the Mayo Clinic converts ED beds into
temporary inpatient beds, making the endpoint for LOS for ED boarders
when they are moved from their ED bed to their assigned ``inpatient''
boarding bed within the ED
\footnote{The Mayo Clinic hospital operations team views ED crowding and boarding as a hospital-wide problem and not an ``ED problem," and they have elected to convert multiple sites around the hospital including pre-operative areas into boarding areas during hospital overcapacity instead of the ED, assigning ED to be the location of boarders as a true last resort.}.

To improve power in our analyses, we drop encounters with rare reasons
for visit (defined as those with fewer than 1,000 total encounters) and
complaints for which a batch order occurs less than 5\% of the time
across all patients, or for which no imaging is ordered. Since batch
orders are rare in these cases, our physician batch tendency instrument
could suffer from a weak-instrument problem if we included them.
Examples of complaints dropped include skin and urinary complaints, as
well as other complaints in which multiple imaging modalities are
unlikely to be indicated. Figure A2 provides a CONSORT-style flow
diagram detailing each exclusion step and the corresponding number of
observations removed, and Appendix Table A1 compares characteristics of
excluded versus included encounters.

Excluding these conditions does not introduce selection bias: Mayo
Clinic's random assignment ensures physicians see the full spectrum of
acuity and complaints regardless of our sample restrictions. We focus on
encounters where the batching decision is both consequential and
discretionary---precisely the population where our LATE provides
actionable policy guidance. While our analytical sample represents 24\%
of total ED encounters, these cases account for 41\% of imaging resource
utilization, underscoring their operational significance. Finally, to
estimate a precise measure of physician-level batch tendency, we
restrict our sample to encounters involving only full-time physicians.
Our final sample includes 11,679 encounters, with chief complaints from
the following categories: Neurological Issue, Abdominal Complaints,
Fevers/Sweats/Chills, Falls/Motor Vehicle Crashes/Assaults/Trauma,
Dizziness/Lightheadedness/Syncope, Extremity Complaints, and
Fatigue/Weakness.

\subsubsection{Treatment Variable}\label{treatment-variable}

Our treatment variable, \(Batched_{i,t}\), is an indicator that equals
\(1\) if patient \(i\) had their tests batch-ordered during their ED
encounter on date \(t\), and \(0\) otherwise. Batching occurs when a
physician simultaneously orders a comprehensive set of diagnostic tests,
typically covering a broad range of potential diagnoses. This contrasts
with standard practice, where a single test is ordered, and subsequent
tests are ordered in sequence as needed.

We define ``batching'' in line with standard emergency medicine
practices and focus on batches that include two or more different
imaging modalities, with the time between orders within 5 minutes, and
that occur as the first imaging tests ordered for the patient encounter
\citep[\citet{jameson2024variation}]{su2025crisis}. We focus on batches
that concern the first imaging tests ordered during the patient
encounter, because this represents the moment of maximum diagnostic
uncertainty---physicians cannot know ex ante which patients will
ultimately require multiple tests, making early batching a discretionary
choice based on practice style rather than clinical necessity. Each
imaging modality, such as X-ray, contrast CT scan, non-contrast CT, and
ultrasound, is considered a separate and distinct test for our study. In
particular, we focus on batching instances in which the physician orders
different imaging tests, as these cannot be performed in a single
scanning session (due to differences in equipment and settings).
Encounters where a single test precedes subsequent batched tests (1.91\%
of multi-test cases) are classified as sequential in our primary
analysis, as the physician has initiated sequential information
gathering before placing additional orders. Sensitivity analyses
conducted around this time window, batch size threshold, and the timing
of the batch show that our results are robust to variations in these
values.

\subsubsection{Dependent Variables}\label{dependent-variables}

The primary outcomes of interest are the ED's efficiency and
effectiveness measures. We measure patient length of stay in two ways.
First, we measure the time from patient arrival until the attending
physician completes care and determines disposition, capturing the
duration until a decision is made to admit, discharge, or transfer the
patient \citep{Feizi2023}. This metric excludes delays related to
inpatient bed availability, providing a more transparent measure of ED
operational efficiency. Second, we measure a patient's total time in the
ED from arrival until physical departure \citep{Lim2024}. This total
time for admitted patients includes the duration until transfer to an
``inpatient'' bed, whether in the main hospital or designated ED areas
converted for inpatient use, encompassing boarding time and discharge
processing \citep{Feizi2023}. Given the documented right-skewness of ED
time metrics \citep{Song2015}, we log-transform both time measurements
to approximate normality \citep[\citet{Saghafian2024}]{Brown2005},
meeting the assumptions required for our regression analyses. As a
robustness check, we also examine treatment time (from physician contact
to disposition), excluding both waiting room delays and post-disposition
boarding; results are qualitatively unchanged and are reported in
Appendix D.

Beyond time-based metrics, we examine resource utilization through the
number of distinct imaging tests performed during each ED encounter.
This count variable captures tests actually completed with documented
results, not merely ordered---an important distinction since ordered
tests may occasionally be cancelled before
completion\footnote{Test cancellations after ordering are extremely rare and face substantial operational barriers. Once the radiology department acknowledges an order, cancellation requires physicians to call and request that the department ``push back" the imaging order. Furthermore, radiology departments often coordinate between modalities (e.g., CT and ultrasound) so patients move directly from one scanner to another.}.
This measure helps us understand how batch ordering practices influence
diagnostic workload.

To assess care quality, we track whether patients return to the ED
within 72 hours of their initial visit and require hospital admission
\citep{Lerman1987}. We focus on returns requiring admission rather than
any 72-hour return because this measure better captures actual quality
failures. Some ED revisits are planned or expected---patients may be
instructed to return for wound checks, suture removal, or if symptoms
persist after initial treatment. Returns requiring admission, however,
are more likely to signal potential issues with initial treatment
decisions, premature discharges, or missed diagnoses. For patients
admitted during their index visit, the 72-hour window begins at hospital
discharge rather than ED departure, ensuring fair comparison across
disposition types. We verify robustness to the broader any-return
measure in Appendix D.

These measures allow us to evaluate ED performance across three critical
dimensions: operational efficiency through time-based measurements,
resource utilization via imaging tests performed, and care quality
through return visit patterns. By examining these outcomes together, we
can assess how batching behaviors affect the efficiency and
effectiveness of care delivery.

\subsection{Identification Strategy}\label{sec:identification}

Our empirical strategy closely follows the literature that relies on the
quasi-random assignment of agents to cases, often referred to as the
``judges design.'' Papers in this literature typically exploit variation
in judges' sentencing leniency within the same court. Similarly, we
explore variation in batching across physicians in the same ED using a
measure we term ``batch tendency.'' We use each physician's residualized
leave-out average batch rate to measure physician batch tendency. We use
this residualized measure of physician batch tendency because, if
certain physicians are more likely to work afternoon or weekend shifts
(as Figure A1 shows are the busiest shifts), the simple leave-out mean
will be biased. A residualized measure of physician batch tendency
accounts for this potential selection. This measure is derived from two
steps following a similar approach used in other applications (e.g.,
\citet{doyle2015measuring}, \citet{dobbie2018effects}, and
\citet{eichmeyer2022pathways}). First, we obtain residuals from a
regression model, which includes all ED encounters in our sample period:

\begin{equation}
Batched_{i,t} = \alpha_0 + \alpha_{ym} + \alpha_{dt} + \mathbf{\beta X_{i,t}} + \varepsilon_{i,t},
\end{equation}

where \(Batched_{i,t}\) is a dummy variable equal to one if patient
\(i\) had their imaging tests batch ordered on an encounter on date
\(t\). Fixed effects include year-month fixed effects, \(\alpha_{ym}\),
to control for time- and season-specific variation in batching,
hospital-specific policies (e.g., initiatives to eliminate excess
testing during a flu season), and seasonality in ED visits. We also
control for ``shift-level'' variations that include both physician
scheduling and patient arrival with day of week-time of day fixed
effects,
\(\alpha_{dt}\)\footnote{Day of week takes on seven values: Sunday, Monday, etc., and time of day are six mutually exclusive four-hour bins: 8 am–12 pm, 12 pm–4 pm, etc.}.
A vector of patient characteristics, \(\mathbf{X_{i,t}}\), including
chief complaint by ESI, vital signs, age, race, and sex, was included to
increase precision. Our primary specification adds further precision
controls, including laboratory tests ordered, physician characteristics,
and ED capacity, as detailed in Section 3.4. As stated in \ref{sec:3.1},
these controls are more than required for our quasi-random assignment
assumption. Under the assumption that we have captured the observables
under which quasi-random assignment occurs in the ED, the unexplained
variation---the physician's contribution---resides in the error term,
\(\varepsilon_{i,t}\).

In step two, the tendency measure for patient \(i\) seen by physician
\(j\) is computed as the average residual across all other patients seen
by the physician during the study period:

\begin{equation}
Batch Tendency_{i,j}^{phys} =
\frac{1}{N_{-i,j}} \sum_{i' \in \{\mathbb{J} \backslash i\}}\hat{\varepsilon}_{i'}
\end{equation}

where \(\hat{\varepsilon}_{i'} = \hat{Batch}_{i'} - Batch_{i'}\) is the
residual from Eq. (1); \(\mathbb{J}\) is the set of all ED encounters
treated by physician \(j\); and
\(N_{-i,j} = |\{\mathbb{J} \backslash i\}|\), the number of cases that
physician has seen that year, excluding patient \(i\). This leave-out
mean eliminates the mechanical bias that arises when patient \(i\)'s
case enters the instrument. The measure is interpreted as the average
(leave-out) batch rate of patient \(i\)'s physician relative to other
physicians in that hospital-year-month, hospital-day of week, and time
of day.

Figure \ref{fig:batch_tendency} provides empirical verification that,
while the decision to batch depends on patient characteristics, our
measure---batch tendency---is plausibly exogenous. The left panel uses a
linear probability model to test whether encounter, patient, ED, and
physician characteristics predict the batching decision, controlling for
shift-level fixed effects. As expected, patient characteristics strongly
predict batching decisions; for instance, patients with
Falls/Assaults/Trauma complaints are 16.2 percentage points more likely
to be batched compared to similar patients under similar ED capacity.
The right panel assesses whether these same characteristics predict
assignment to physicians with different batch tendencies. Importantly,
we find that patient characteristics do not significantly predict
assignment to high- or low-batch-tendency physicians. The coefficients
are near zero, with confidence intervals crossing zero for all patient
characteristics, confirming that, conditional on shift-fixed effects
(which account for mechanical rotation), the assignment of patients to
physicians with different batching tendencies is effectively random.
This validates the rotational assignment mechanism and establishes batch
tendency as an exogenous source of variation for identifying causal
effects.

\begin{figure}[t!]
\caption{Batch Tendency by Patient Characteristics}
\includegraphics[width=\textwidth]{../outputs/figures/fig2_panel_batched_standardized.png}
\label{fig:batch_tendency}
\begin{flushleft}
\footnotesize \textit{Notes:} This figure tests quasi-random assignment of patients to physicians in the Mayo Clinic ED. The left panel shows how patient characteristics predict batching decisions. The right panel shows that these same characteristics do not predict assignment to physicians with different batch tendencies.
\end{flushleft}
\end{figure}

We further document that (a) there is significant variation in this
measure, and (b) the measure is highly predictive of the decision to
batch. Figure \ref{fig: tendency} shows the distribution of physician
batch tendency and the relationship between batch tendency and batching,
where the relationship is illustrated via local linear regression of
batching against physician tendency. As shown, the probability of
batching increases approximately linearly and monotonically with our
tendency measure (see Table \ref{tab:first_stage} for more formal
results).

\begin{figure}[t!]
\begin{center}
\caption{Distribution and First Stage of Instrument}
\includegraphics[width=0.65\textwidth]{../outputs/figures/Fig3_firststage.png}
\label{fig:tendency}
\end{center}
\begin{flushleft}
\footnotesize \textit{Notes:} This figure plots the histogram of physician batch tendency along the x-axis and the left y-axis for all patient encounters. A local-linear regression of the fitted probability of batching on batch tendency, after residualizing (see text for baseline fixed effects and controls in residualization), is overlaid on the right y-axis. Ninety-five percent confidence bands are also shown.
\end{flushleft}
\end{figure}

Together, we observe that batch tendency likely meets the criteria for a
valid IV. In the next section, we formalize our IV analysis questions
and more formally establish the validity of our IV.

\subsubsection{IV Analysis}\label{iv-analysis}

To estimate the reduced-form effects of being treated by a
batch-preferring physician (batcher), we estimate the following
equation:

\begin{equation}
Y_i = \mu_0 + \mu_1 \, BatchTendency_{i,j}^{phys} + \mathbf{\gamma X_i} + \nu_i
\end{equation}

To study the effects of test batching on outcomes, \(Y_i\), we estimate
the following Two-Stage Least Squares (2SLS) equations:

\begin{equation}
\left\{
\begin{aligned}
\text{First Stage:} \quad & Batched_i = \delta_0 + \delta_1 \, BatchTendency_{i,j}^{phys} + \mathbf{\delta_2 X_i} + \nu_i \\
\text{Second Stage:} \quad & Y_i = \beta_0 + \beta_1 \hat{Batched_i} + \mathbf{\theta X_i} + \varepsilon_i
\end{aligned}
\right.
\end{equation}

In all cases, \(Y_i\) represents our primary outcomes of interest, and
\(\mathbf{X_i}\) includes the same covariates as in Eq. 1 and additional
controls for physician experience, physician sex, and ED capacity level
(ED capacity guidelines for Mayo Clinic are in Appendix Table A1. The
variable \(Batched_i\) may be endogenous; for example, injury severity
may be unobserved and correlated with the need to run multiple tests,
length of stay, and the likelihood of returning with admission. Hence,
we instrument \(Batched_i\) with the assigned physician \(j\)'s
underlying tendency to batch, \(BatchTendency_{i,j}^{phys}\). We cluster
robust standard errors at the physician level to account for the
assignment process of patients to physicians.

Table \ref{tab:first_stage} presents formal first-stage results from Eq.
(4). Column 1 of Table 2 presents the mean crude batch rate. Column 2
reports results only with year-month and day-of-week-time-of-day fixed
effects. Column 3 adds our baseline patient and hospital condition
controls. Consistent with Figure \ref{fig:tendency}, our residualized
physician instrument is highly predictive of whether a patient will have
their imaging tests batch ordered. Including controls in column 3 does
not change the magnitude of the estimated first-stage effect, consistent
with the quasi-randomness of patients to physicians with different
batching tendencies.

Furthermore, the batch tendency measure reasonably predicts the batching
decision, and the IV is not weak (F-stat = 171.9). For example, across
all controls (column 3), our results show that a patient assigned to a
physician at the 90th percentile of batch tendency (0.023) relative to a
physician at the 10th percentile (-0.024) is 8.2 percentage points more
likely to have their tests batched. The coefficient is greater than one
because all emergency visits are used to construct the tendency
instrument. The first stage is calculated using only the baseline
sample, excluding the rare and rarely batched complaints.

\begin{table}[t!]
\centering
\caption{First-Stage Results: Batch Tendency and Batching}
\label{tab:first_stage}
\begin{threeparttable}
\begin{tabular}{p{8cm}ccc}
\toprule
 & Sample Mean & \multicolumn{2}{c}{Batched} \\
\cmidrule(lr){3-4}
 & (1) & (2) & (3) \\
\midrule
Batch Tendency & 0.137 & $1.837^{***}$ & $1.752^{***}$ \\
 & (0.343) & (0.138) & (0.134) \\
\midrule
\textit{Controls} \\
Necessary controls & --- & Yes & Yes \\
Precision controls & --- & No & Yes \\
\midrule
Adj. $R^2$ & --- & 0.022 & 0.74 \\
F-stat & --- & 171.9 & 171.9 \\
Observations & 11,679 & 11,679 & 11,679 \\
\bottomrule
\end{tabular}
\begin{flushleft}
\footnotesize
\textit{Notes:} This table reports first-stage results for the regression of batch tendency on the likelihood of batching. Column 1 reports the dependent variable means for patients managed according to standard practice. Column 2 includes quasi-random assignment necessary controls for year-month and day of week-time of day. Column 3 adds baseline precision controls, including patient characteristics (age, ESI-complaint, race, sex, and vital signs such as tachycardia, tachypnea, febrile status, and hypotensive status), laboratory tests ordered, ED capacity level, and physician characteristics (physician sex, experience, and hours into shift). Robust standard errors are heteroskedasticity-robust.
$^{***} p < 0.01$.
\end{flushleft}
\end{threeparttable}
\end{table}

\subsubsection{UJIVE Construction}\label{ujive-construction}

Leniency designs like ours, where many providers generate the
identifying variation, can suffer from many-instrument bias in
conventional 2SLS: the constructed first-stage fitted values
mechanically reuse each observation's own treatment status, creating a
small-sample correlation between the instrument and the structural error
\citep{kolesar2015manyinvalid, goldsmithpinkham2025leniencydesignsoperatorsmanual}.
This bias can be substantial even when first-stage F-statistics appear
adequate.

To address this concern, we implement the Unbiased Jackknife
Instrumental Variables Estimator (UJIVE) as a robustness check for our
main 2SLS specification. While our primary design uses a single
continuous instrument---physician batch tendency---UJIVE uses the full
set of physician indicators, \(Z_j\), as instruments. For each
observation \(i\) treated by physician \(j\), UJIVE constructs a
leave-one-out predicted instrument:

\begin{equation}
\hat{\ell}_{i,-i} = Z_j' \hat{\pi}_{-i}
\end{equation}

where \(\hat{\pi}_{-i}\) is estimated from the first-stage regression of
\(Batched\) on physician indicators using all observations except \(i\).
Because \(\hat{\pi}_{-i}\) excludes observation \(i\)'s own treatment
status, the predicted instrument \(\hat{\ell}_{i,-i}\) is orthogonal to
the structural error by construction. This eliminates the mechanical
correlation that contaminates conventional many-instrument 2SLS while
preserving consistency for the local average treatment effect.

This estimator is asymptotically unbiased in many-instrument leniency
designs and yields valid standard errors even when conventional 2SLS
understates uncertainty. If our 2SLS results were driven by
many-instrument bias, we expect UJIVE estimates to diverge
substantially. As we show in Section 4, UJIVE estimates closely track
our 2SLS results across all outcomes, increasing confidence that
many-instrument bias is not driving our findings.

\subsubsection{Identifying
Assumptions}\label{sec:identifying_assumptions}

The reduced-form approach delivers an unbiased estimate of the causal
effect of being treated by a physician with a higher tendency to batch,
since patient assignment to ED physicians is random and conditional on
seasonality and shift (``conditional independence''). The
residualization in Eq. (1) allows for further statistical precision in
measuring the physician's tendency to batch.

Our instrumental variable approach, which aims to recover the causal
effect of batch ordering diagnostic tests, relies on three additional
assumptions: relevance, exclusion, and monotonicity. We reported a
strong first stage (i.e., relevance) at the end of the previous section.
The exclusion restriction requires the instrument to influence the
outcome of interest only through its effect on test batching. This
assumption is untestable, but we take it seriously and address it
through multiple strategies. We expand our precision control set to
include laboratory tests ordered, physician characteristics (experience,
gender, hours into shift), and ED capacity levels in addition to patient
complaint, sex, race, acuity, and vital signs. Additionally, we test
this assumption by performing a placebo check for rarely batched
complaints (Appendix D) and various robustness checks in Section
\ref{sec:robustness}, including sensitivity analyses that allow for
plausible violations of the exclusion restriction
\citep{conley2012plausibly}.

To the extent that residual exclusion restriction concerns remain, our
reduced-form estimates can be interpreted as the causal effect of being
assigned to a high-versus low-batch-tendency physician---a
policy-relevant parameter for ED managers considering interventions
around physician feedback or training.

Finally, the monotonicity assumption is necessary for interpreting the
IV coefficient estimates as LATEs when there are heterogeneous treatment
effects. It requires that any patient who a low-batch-tendency physician
batches be batched by a high-batch-tendency physician. The literature
that leverages the judges' design typically performs two informal tests
to assess its implications. The first provides that the first stage
should be weakly positive for all subsamples \citep{dobbie2018effects}.
The second asserts that the instrument constructed by omitting a
particular subsample has predictive power for that same subsample
\citep{bhuller2020incarceration}. Appendix Figure D1 presents both of
these tests for various subsamples of interest. In the left panel, our
residualized measure of batch tendency is consistently positive and
sizable across all subsamples, consistent with the monotonicity
assumption. In the right panel, we also find that our additional
first-stage results are consistently same-signed and sizable across all
subsamples. The coefficient magnitudes differ across subgroups because
rates of batching differ.

\subsubsection{More Details on Our LATE
Estimates}\label{more-details-on-our-late-estimates}

Our two-stage least squares estimates represent the LATE of batch
ordering for `compliers'---patients whose testing strategy depends on
the assigned physician's practice style. This effect compares early
batching to standard practice, which includes both sequential ordering
and single tests. While this involves a composite counterfactual, it
provides the policy-relevant parameter: the effect of encouraging
comprehensive upfront testing versus allowing diagnostic information to
guide testing decisions for patients at the margin of clinical
discretion---those whose testing strategy is not dictated by apparent
clinical necessity but rather depends on physician practice style and
judgment. To better understand this LATE, we characterize the number of
compliers and their characteristics following the approach developed by
\citet{abadie2003economic} and extended by \citet{dahl2014peer}.

Specifically, compliers are defined as patients whose batched status
depends on whether their physician has the highest batch tendency
(\(\bar{z}\)) or the lowest batch tendency (\(\underline{z}\)).
Mathematically, the fraction of compliers (\(\pi_c\)) is given by:

\[
\pi_c = P(Batched | Z = \bar{z}) - P(Batched | Z = \underline{z}),
\]

where \(Z\) represents the batch tendency of the physician. Using the
first-stage regression model, we predict the probabilities of being
batched under the most lenient (\(\bar{z}\)) and strict
(\(\underline{z}\)) physicians.

Approximately 13 percent of patients in our sample are ``compliers,''
meaning they would have received batched tests if assigned to a
high-batch-tendency physician but received standard practice otherwise.
In comparison, 5 percent of patients are ``always takers,'' meaning they
would receive batched tests regardless of the physician's batch
tendency, and 82 percent are ``never takers,'' meaning they would never
receive batched tests regardless of the physician's batch tendency.
Appendix B provides more details on complier estimates and their
characteristics.

\section{Results and Discussion}\label{results-and-discussion}

\subsection{Reduced-Form Results}\label{sec:reducedform}

In this sub-section, we explore the causal influence of physician batch
tendency on patient outcomes and resource utilization in the ED. We find
statistically and operationally significant effects of assignment to a
high-batch-tendency physician on every outcome except 72-hour return
with admission (Table \ref{tab:red_form}). Scaling our coefficients by
the difference in tendency going from the lowest decile to the highest
decile in physician tendency---equal to a 5.0 percentage point
increase---for interpretability, we find that assignment to a physician
in the top batching decile (relative to one in the bottom decile) is
associated with a 5.8\% increase in time to disposition, a 5.3\%
increase in LOS, and an additional 10.9 imaging tests ordered per 100
patient encounters. These findings highlight ED physicians' substantial
role in putting patients on a path toward longer LOS and increased
resource
utilization\footnote{For reduced-form results for a less strict sample see, e.g., \cite{jameson2024variation}, \cite{hodgson2018are}, and the references therein.}

\begin{table}[t!]
\centering
\caption{Reduced Form: Batch Tendency and Patient Outcomes}
\label{tab:red_form}
\begin{threeparttable}
\begin{tabular}{p{5cm}cccc}
\toprule
\multicolumn{5}{c}{Dependent variable} \\
\cmidrule(lr){2-5}
& \makecell[c]{Log time to \\ disposition\\(1)} 
& \makecell[c]{Log \\ LOS\\(2)} 
& \makecell[c]{Number of \\ distinct \\ imaging tests\\(3)} 
& \makecell[c]{72hr return \\ with admission\\(4)} \\
\midrule
Batch tendency  
& $1.140^{***}$ & $1.046^{***}$ & $2.173^{***}$ & $-0.0257$ \\ 
& (0.1592) & (0.1350) & (0.2103) & (0.0341) \\[0.75em]

\makecell[l]{Scaled coeffecient:\\ 10th $\rightarrow$ 90th pct. ($\Delta = 0.050$)}  
& 0.0576 & 0.0525 & 0.1089 & -0.00129 \\[1em]

Mean dependent variable 
& 5.248 & 5.505 & 1.450 & 0.0121 \\ 
& (0.493) & (0.449) & (0.623) & (0.109) \\ 

Necessary controls & Yes & Yes & Yes & Yes \\ 
Precision controls & Yes & Yes & Yes & Yes \\ 
\midrule
Adj.\ $R^2$ & 0.2109 & 0.2834 & 0.1346 & 0.0103 \\ 
Observations & 11{,}679 & 11{,}679 & 11{,}679 & 11{,}679 \\
\bottomrule
\end{tabular}

\begin{flushleft}
\footnotesize
\textit{Notes:} This table reports reduced-form estimates of the relationship between physician batch tendency and patient outcomes. 
The second row rescales coefficients to represent the effect of moving from the 10th to the 90th percentile of batch tendency. 
($\Delta = 0.050$). 
$^{***} p < 0.001$.
\end{flushleft}

\end{threeparttable}
\end{table}

The fact that our primary outcomes respond strongly to physician batch
tendency suggests that batching is the underlying mechanism behind the
effects. However, physicians could differ in other dimensions of
care---some observable and others not---which could be correlated with
batch tendency. Appendix D attempts to distinguish between and identify
the mechanisms behind the observed reduced-form effects. This mediation
analysis provides a crucial step toward a well-identified IV analysis,
which we present in Section 4.2 using both 2SLS and UJIVE estimators.

\subsection{Instrumental Variables
Estimation}\label{instrumental-variables-estimation}

Next, we examine the effects of batch ordering imaging tests using the
IV strategy described above. We first analyze the effects of batching on
primary ED operational outcomes before examining its impacts on specific
test ordering patterns and disposition decisions.

Panel A of Table \ref{tab:results_table} presents 2SLS and UJIVE
estimates of the impact of batching on key operational metrics. Column 1
reports the dependent variable means for patients managed through
standard practice. Columns 2 and 3 report 2SLS estimates using our
physician batch tendency instrument, without and with precision
controls, respectively. Columns 4 and 5 report UJIVE estimates, which
address potential many-instrument bias by using leave-one-out
predictions from the complete set of physician indicators.

\begin{table}[t!]
\centering
\caption{Effect of Batching Tests on Patient Outcomes}
\label{tab:results_table}
\begin{threeparttable}
\begin{tabular}{lccccc}
\toprule
& Sequenced & \multicolumn{2}{c}{\underline{2SLS}} & \multicolumn{2}{c}{\underline{UJIVE}} \\
& mean & (2) & (3) & (4) & (5) \\
\midrule

\multicolumn{6}{l}{\textit{Panel A. Primary Outcomes}} \\[0.5em]

Log time to disposition 
& 5.237 
& $0.659^{***}$ & $0.651^{***}$
& $0.583^{***}$ & $0.522^{***}$ \\
& (0.499)
& (0.103) & (0.101)
& (0.189) & (0.177) \\[0.5em]

Log LOS
& 5.490
& $0.717^{***}$ & $0.597^{***}$
& $0.653^{***}$ & $0.503^{***}$ \\
& (0.456)
& (0.094) & (0.088)
& (0.158) & (0.144) \\[0.5em]

Number of distinct imaging tests 
& 1.335
& $1.385^{***}$ & $1.241^{***}$
& $1.316^{***}$ & $1.174^{***}$ \\
& (0.572)
& (0.118) & (0.116)
& (0.126) & (0.119) \\[0.5em]

72-hour return with admission 
& 0.012
& -0.0137 & -0.0146
& -0.0079 & -0.0039 \\
& (0.110)
& (0.018) & (0.019)
& (0.020) & (0.022) \\[0.5em]

72hr return 
& 0.030
& -0.0512 & -0.0536
& -0.0440 & -0.0396 \\
& (0.170)
& (0.029) & (0.031)
& (0.032) & (0.034) \\[0.5em]


\multicolumn{6}{l}{\textit{Panel B. Test Types}} \\[0.5em]

X-ray 
& 0.576
& $0.943^{***}$ & $0.989^{***}$
& $0.960^{***}$ & $0.959^{***}$ \\
& (0.494)
& (0.100) & (0.101)
& (0.116) & (0.117) \\[0.5em]

Ultrasound 
& 0.171
& $0.160^{**}$ & $0.087$
& $0.164^{*}$ & $0.087$ \\
& (0.377)
& (0.076) & (0.073)
& (0.082) & (0.078) \\[0.5em]

CT without contrast 
& 0.400
& $0.102$ & $0.062$
& $0.052$ & $0.053$ \\
& (0.490)
& (0.095) & (0.086)
& (0.112) & (0.102) \\[0.5em]

CT with contrast 
& 0.187
& $0.180^{*}$ & $0.102$
& $0.140$ & $0.075$ \\
& (0.390)
& (0.078) & (0.076)
& (0.087) & (0.079) \\[0.5em]

\multicolumn{6}{l}{\textit{Panel C. Disposition}} \\[0.5em]
Admission 
& 0.279
& $0.419^{***}$ & $0.404^{***}$
& $0.424^{***}$ & $0.398^{***}$ \\
& (0.449)
& (0.096) & (0.088)
& (0.103) & (0.090) \\[0.5em]
\midrule
Necessary controls & --- & Yes & Yes & Yes & Yes \\
Precision controls & --- & No & Yes & No & Yes \\
Observations & 11,679 & 11,679 & 11,679 & 11,679 & 11,679 \\
\bottomrule
\end{tabular}
\begin{flushleft}
\footnotesize
\textit{Notes:} Column 1 reports means for standard care patients (standard deviations in parentheses). Columns 2–3 report 2SLS results using physician batch tendency as the instrument. Columns 4–5 report UJIVE estimates using ED provider identifiers as many weak instruments. All models include day-of-week and time-of-day fixed effects, as well as month-year fixed effects. Precision controls are described in the text. Standard errors are heteroskedasticity-robust.
\item $^{*} p < 0.05$, $^{**} p < 0.01$, $^{***} p < 0.001$.
\end{flushleft}
\end{threeparttable}
\end{table}

The UJIVE estimates with full controls (Column 5)---our preferred
specification---indicate that discretionary batching substantially
increases ED length of stay. The marginal batched patient experiences a
65\% increase in total ED length of stay and a 69\% increase in time to
disposition compared to patients managed through standard practice.
Notably, the estimates attenuate when precision controls are added
(comparing Columns 2-3 and Columns 4-5), confirming that our original
specification captured some correlation with general diagnostic
intensity. However, large and significant effects persist.

Discretionary batching also leads to more intensive diagnostic testing.
The marginal batched patient receives 1.2 additional imaging tests
(Column 5), representing an 88\% increase from the mean for patients
managed through standard practice. This increased testing intensity does
not appear to improve care quality: effects on 72-hour returns with
admission are small, negative, and statistically indistinguishable from
zero. These findings suggest that discretionary batching results in
additional delay-inducing tests that do not add diagnostic value. Put
differently, standard practice offers an important benefit: the
information obtained from initial tests reduces the need for subsequent,
non-value-adding but delay-inducing tests (an ``information gain''
advantage).

Panel B of Table \ref{tab:results_table} examines how batching affects
the utilization of specific imaging modalities. The estimates reveal
that batching leads to significant increases in X-ray utilization: with
full controls, batched patients are approximately 96 percentage points
more likely to receive an X-ray (\(p<0.001\)). Effects on other
modalities---ultrasound, CT without contrast, and CT with contrast---are
smaller and not statistically significant after adding precision
controls. This pattern suggests that physicians who batch construct
comprehensive workups primarily by adding quick, low-cost imaging
(X-rays) rather than selectively ordering expensive advanced imaging.

Panel C of Table \ref{tab:results_table} presents estimates of
batching's impact on disposition decisions. The UJIVE results with full
controls indicate that discretionary batching increases the admission
probability by 40 percentage points. This also means that the impact of
batching in the EDs likely spills over to other parts of hospitals
(e.g., inpatient units), increasing their patient volumes. Increased
patient volume, in turn, is known to induce behaviors such as speed-up,
which can harm other quality-of-care metrics, such as 30-day mortality
\citep{Song2019}.

These results paint a consistent picture of the operational implications
of batching. Discretionary batching results in more comprehensive
diagnostic workups but also significantly longer processing times and a
higher admission probability, without measurable improvements in
short-term quality outcomes. The stability of estimates across 2SLS and
UJIVE specifications, and the expected minor attenuation when adding
precision controls, strengthens confidence that these findings reflect
the causal effect of imaging-specific batching behavior rather than
general physician diagnostic style.

\subsection{Potential Mechanism: Mediation
Analysis}\label{potential-mechanism-mediation-analysis}

To better understand the mechanisms by which batching in the ED affects
key operational outcomes---LOS and time to disposition---we investigated
the mediating roles of imaging volume (number of tests) and admission
decisions. Separate analyses were conducted for each operational outcome
to ensure a comprehensive assessment of these pathways. These variables
were selected as potential mediators for two reasons: (a) our prior
analyses demonstrate that batching significantly increases both the
number of tests ordered and the likelihood of admission, and (b) there
is a strong theoretical basis to believe that these variables contribute
to longer LOS and time to disposition. Increased imaging volume can
delay patient processing due to the time required to conduct and
interpret diagnostic tests. Similarly, resource utilization and
diagnostic intensity are known to influence admission decisions,
directly impacting ED operational metrics by affecting patients who
could be discharged but are admitted \citep{hodgson2018are}.

We formalized our hypothesized mechanisms using structural equation
modeling (SEM) and the underlying directed acyclic graphs (DAGs)
depicted in Figure \ref{fig:dag}. Parts A and B of this figure represent
the underlying DAGs when LOS and time to disposition are considered the
primary outcomes of interest, respectively.
\footnote{Of note, a change in admission decision (discharge vs. admit) in the ED can influence a patient's LOS (part A) but not their time to disposition (part B), since the latter by definition excludes delays after the disposition decision.}
SEM simultaneously estimates multiple interrelated regression equations,
explicitly modeling direct and indirect effects through causal pathways
identified in a DAG. This approach enables a quantitative evaluation of
the theoretical model in which batching influences operational outcomes
both directly and indirectly via mediators
\citep[\citet{Zhang2017}]{Beran2010}. In our SEM analyses, to control
for potential confounding due to time and patient complexity, we first
residualized all variables of interest by regressing them on fixed
effects, including day of the week, month, and chief complaint severity.
This residualization ensures that our estimates reflect associations net
of these confounders, enabling a more precise exploration of the
relationships among batching tendency, mediators, and outcomes.

\begin{figure}[t!]
\centering
\includegraphics[width=\textwidth]{figures/dag.png}
\caption{Directed Acyclic Graph (DAG) of the Mediation Analysis}
\label{fig:dag}
\end{figure}

The mediation analysis results, detailed in Appendix Tables C1 and C2,
shed light on a plausible mechanism by which batching impacts LOS and
time to disposition. Our findings suggest that the effects of batching
are primarily mediated through increased diagnostic testing intensity.
This aligns with the results in Table \ref{tab:results_table},
demonstrating that batching increases imaging volume. This pathway
accounts for substantial delays in LOS, as additional imaging requires
time for completion, interpretation, and integration into clinical
decision-making. For LOS (Figure \ref{fig:dag} part A), the indirect
effect via imaging volume was estimated at \(0.207\) (\(p < 0.001\)),
indicating a substantial contribution to the total delay observed.
Similarly, for time to disposition (Figure \ref{fig:dag} part B), the
indirect effect via imaging was \(0.085\) (\(p < 0.001\)), underscoring
the central role of diagnostic testing intensity in extending processing
times. In contrast, the direct effects of batching on both LOS and time
to disposition were small and statistically insignificant, as shown in
Appendix Tables C1 and C2. This suggests that the delays attributed to
batching are driven not by the act of batching itself, but by the
subsequent increase in diagnostic testing activity that it entails.

For LOS (Figure \ref{fig:dag} part A), the role of admission decisions
adds complexity to this narrative. Affirming the results in Table
\ref{tab:results_table}, the SEM analysis (Table C1) revealed that
batching was associated with a significant increase in admission
likelihood (\(b_2 = 0.327\), \(p < 0.001\)). This suggests that
increased imaging may contribute to more conservative disposition
decisions, potentially due to incidental findings or heightened
diagnostic uncertainty, which necessitate further inpatient care. This
pathway contributed to additional delays in LOS, as coordinating patient
transfers from the ED to inpatient units involves logistical challenges
and often requires additional time due to bed availability. The total
indirect effect for LOS, combining the pathways through imaging tests
and admissions, was estimated at \(0.256\) (\(p < 0.001\)), while the
direct effect of batching remained insignificant (\(c^\prime = 0.031\),
\(p = 0.662\)). This consistency between the 2SLS and SEM results
(Tables \ref{tab:results_table}, C1, and C2) reinforces the conclusion
that diagnostic intensity and admission are the key drivers of increased
LOS among batched patients.

For time-to-disposition, where admission is not a mediator, the indirect
effect through imaging tests (\(0.085\), \(p < 0.001\)) remains the
predominant pathway, with the direct effect of batching again
nonsignificant (\(c^\prime = -0.100\), \(p = 0.216\)). This further
highlights that the perceived efficiency gains from batching outweigh
the operational burdens associated with increased diagnostic testing.

\subsection{Heterogeneous Effects: ED Capacity
Status}\label{heterogeneous-effects-ed-capacity-status}

Given that ED capacity constraints significantly influence operational
decisions, we examine whether the effects of batching vary across
different capacity levels. Following the Mayo Clinic ED's internal
guidelines\footnote{The ED of our other partner hospital (MGH) does not follow these guidelines. Hence, we did not include these analyses in our study of the effects of batching at MGH (see Section \ref{sec:generalize} for our analysis using MGH data).}
(Table A1), we categorize each encounter into three categories:
occurring during normal operations, minor overcapacity, or major
overcapacity. While our primary analysis controls for capacity status,
stratifying by this variable allows us to assess Hypothesis 4---whether
the effects of batching are attenuated under severe resource
constraints.

Table \ref{tab:het_effects} presents 2SLS estimates stratified by ED
capacity status. We observe that batching rates decline modestly as
capacity constraints intensify: from 14.8\% during normal operations to
12.7\% under minor overcapacity and 12.5\% under major overcapacity.
This pattern suggests that physicians become somewhat more selective in
their batching decisions when facing resource constraints.

\begin{table}[t!]
\centering
\caption{Effects of Batching by ED Capacity Status}
\label{tab:het_effects}
\begin{threeparttable}
\begin{tabular}{lccc}
\toprule
& Normal & Minor & Major \\ 
& Operations & Overcapacity & Overcapacity \\ 
& (1) & (2) & (3) \\ 
\midrule

Log LOS 
& $0.603^{***}$ & $0.645^{***}$ & 0.309 \\ 
& (0.119) & (0.155) & (0.262) \\[0.75em] 

Log time to disposition 
& $0.638^{***}$ & $0.696^{***}$ & 0.404 \\ 
& (0.138) & (0.178) & (0.276) \\[0.75em] 

Number of distinct imaging tests 
& $1.201^{***}$ & $1.171^{***}$ & $1.914^{***}$ \\ 
& (0.151) & (0.203) & (0.479) \\[0.75em] 

72-hour return with admission 
& $-0.028$ & 0.006 & $-0.020$ \\ 
& (0.026) & (0.033) & (0.073) \\[0.75em] 
\midrule
Batch rate & 0.148 & 0.127 & 0.125 \\[0.5em]
\\
Observations 
& 5,241 & 5,355 & 1,083 \\ 

Necessary controls & Yes & Yes & Yes \\ 
Precision controls & Yes & Yes & Yes \\ 
\bottomrule
\end{tabular}

\begin{flushleft}
\footnotesize
\textit{Notes:} This table reports two-stage least squares estimates of the impact of batching across different ED capacity levels. ED capacity levels are defined in accordance with internal guidelines in Appendix Table A1. All specifications include time fixed effects and baseline controls. 
$^{***}p<0.001$.
\end{flushleft}
\end{threeparttable}
\end{table}

However, we do not find statistically significant differences in the
effects of batching across capacity levels. During normal operations,
batching increases LOS by approximately 83\% (Column 1), while under
minor overcapacity, the effect is similar at 90\% (Column 2). Given
major overcapacity, the point estimate is smaller (36\%) and not
statistically significant. However, the confidence intervals are wide
due to the smaller sample size (n=1,083), and we cannot reject the
hypothesis that the effect equals that observed under normal operations.
Similarly, imaging volume effects remain large and significant across
all capacity levels, ranging from 1.17 to 1.91 additional tests per
batched patient.

These results do not support Hypothesis 4. While batching rates decline
modestly under overcapacity---suggesting physicians exercise somewhat
greater selectivity---the estimated effects of batching on outcomes
remain statistically indistinguishable across capacity levels. The
operational inefficiencies associated with discretionary batching
persist regardless of ED utilization. From a policy perspective, this
suggests that interventions to reduce unnecessary batching are warranted
across all operational contexts, not only during periods of normal
capacity.

\subsection{Determinants of Image
Batching}\label{determinants-of-image-batching}

To investigate the drivers of batching and image ordering behavior, we
examine the relationship between physician characteristics, ED crowding,
and the likelihood of batched testing. We estimate the following
regression models:

\small

\begin{equation}
Y_{i,j} = \beta_0 + \mathbf{\beta_1 MD_j} + \gamma Capacity + \mathbf{\alpha X_{i}} + \epsilon_i
\end{equation} \normalsize

Where \(Y_{i,j}\) represents our outcome of interest: Batched, a binary
measure of whether physician \(j\) batched tests for patient \(i\), and
the number of images ordered for patient \(i\) by physician \(j\).
\(\mathbf{MD_j}\) is a vector of physician characteristics, including
years since residency graduation, whether the physician is male, and the
number of hours they are into their shift. \(Capacity\) is the current
capacity level of the ED, defined in Appendix Table A1.
\(\mathbf{X_{i}}\) is the vector of patient covariates described in the
previous section and in Figure \ref{fig:batch_tendency}. We cluster
robust standard errors at the physician level. Table
\ref{tab:determinants} presents the results.

\begin{table}[t!]
\centering
\caption{\textbf{Determinants of Test Ordering Behavior}}
\label{tab:determinants}
\begin{threeparttable}
\begin{tabular}{p{7.5cm}cc}
\toprule
& Batched & Number of Imaging Tests \\
& (1) & (2) \\ 
\midrule

\multicolumn{3}{l}{\textit{Panel A. Physician Characteristics}} \\[0.5em]

Physician experience (years since residency)
    & $-0.001^{*}$ 
    & $-0.003^{***}$ \\
& (0.000) 
& (0.001) \\[0.75em]
Physician male 
    & 0.006 
    & 0.008 \\
& (0.007) 
& (0.013) \\[0.75em]
Hours into shift 
    & $-0.004^{*}$ 
    & 0.000 \\
& (0.001) 
& (0.003) \\[0.75em]
\multicolumn{3}{l}{\textit{Panel B. ED Conditions}} \\[0.5em]
Capacity Level: Minor overcapacity 
    & $-0.017^{**}$ 
    & $-0.026^{*}$ \\
& (0.007) 
& (0.011) \\[0.75em]
Capacity Level: Major overcapacity 
    & $-0.019$ 
    & -0.017 \\
& (0.011) 
& (0.019) \\[0.75em]
\midrule
Necessary controls & Yes & Yes \\
Precision controls & Yes & Yes \\
Observations & 11{,}679 & 11{,}679 \\
$R^2$ & 0.051 & 0.121 \\
\bottomrule
\end{tabular}
\begin{flushleft}
\footnotesize
\textit{Notes:} Table reports OLS estimates of relationships between physician/ED characteristics and test-ordering behavior. 
All models include FE for shift and patient characteristics as described in the text.
$^{*}p<0.05$, $^{**}p<0.01$, $^{***}p<0.001$.
\end{flushleft}
\end{threeparttable}
\end{table}

We find that physician experience is associated with modest reductions
in both batching and overall imaging volume. Each additional year since
residency is associated with a 0.1 percentage-point decrease in the
probability of batching (\(p<0.05\)) and 0.3 fewer imaging tests per 100
encounters (\(p<0.001\)). Physician gender shows no significant
relationship with either batching or test volume.

The timing within a physician's shift significantly influences batching
decisions. For each additional hour into the shift, the likelihood of
batching decreases by 0.4 percentage points (\(p<0.05\)), though hours
into the shift show no significant relationship with overall imaging
volume. Because the Mayo Clinic ED features very few handoffs and
physicians tend to stay with their patients until disposition, this
decline in batching as shifts progress may reflect physicians adopting
more conservative testing strategies to ensure they can complete their
work before their shift ends.

ED capacity conditions also influence test ordering patterns. Under
minor overcapacity, we observe significant decreases in both batching
(1.7 percentage points, \(p<0.01\)) and imaging volume (2.6 fewer tests
per 100 encounters, \(p<0.05\)). Given major overcapacity, point
estimates suggest further reductions, but these effects are not
statistically significant---likely due to the smaller number of
encounters during these periods (n=1,083). This pattern is consistent
with physicians becoming more selective in their ordering behavior when
facing resource constraints, though the relationship is modest in
magnitude.

\subsection{Generalizability of Results Across
EDs}\label{sec:generalize}

To assess the generalizability of our findings beyond the Mayo Clinic
ED, we replicated our analysis using data from the MGH ED, one of the
busiest emergency departments in the United States. The MGH dataset
comprises 129,489 patient encounters from November 10, 2021, through
December 10, 2022. This extensive dataset provides a robust sample to
validate the external applicability of our results.

Unlike the Mayo Clinic ED, where patients are randomly assigned to
physicians upon arrival through a rotational system, the MGH ED employs
a different patient assignment mechanism. At MGH, patients are triaged
into different care areas (e.g., urgent care, fast track, observation)
based on acuity and presenting complaints, then assigned to physicians
within those areas based on availability rather than through random
rotation. To address this non-random assignment and potential selection
bias, we adjust our instrumental variable strategy by including
additional covariates for care area assignment, acuity level, and
presenting complaints in both stages of our 2SLS and instrument
construction, thereby accounting for the sorting of patients into
different ED zones. While this approach cannot guarantee the same level
of causal identification as Mayo's randomized system, it provides a more
robust comparison of the effects of batching on patient outcomes across
different ED settings.

After adjusting for institutional differences and using the same
exclusion criteria we used with Mayo, we find strong evidence that our
key findings generalize to the MGH setting. The 2SLS results in Table
\ref{tab:generalize} suggest that batching leads to a 44.3\% increase in
length of stay and approximately 1.8 additional imaging tests per
patient. To formally assess whether the estimated effects differ
significantly across the two ED settings, we conduct a Z-test comparing
the 2SLS coefficients from MGH and Mayo by estimating:

\[
Z = \frac{\hat{\beta}{\text{MGH}} - \hat{\beta}{\text{Mayo}}}{\sqrt{SE_{\text{MGH}}^2 + SE_{\text{Mayo}}^2}}
\].

As reported in Column 4, the Z-statistics for each outcome indicate no
statistically significant differences in the estimated effects between
the two settings. This suggests that the impact of batching is mainly
consistent across hospitals despite differences in patient assignment
mechanisms and operational structures. Overall, these results reinforce
the external validity of our findings and provide further evidence that
the observed effects of batching are not merely an artifact of a single
institution's workflow but a systematic consequence of batching in
high-volume emergency care settings.

\begin{table}[t!]
\centering
\caption{\textbf{Comparison of Effects of Batching Across Hospital Settings}}
\label{tab:generalize}
\begin{threeparttable}
\begin{tabular}{p{6cm}cccc}
\toprule
& \makecell{Sequenced \\ Mean (SD)} & OLS & 2SLS & \makecell{Z-Statistic} \\
& (1) & (2) & (3) & (4) \\
\midrule
Log LOS & 6.42 & $0.21^{***}$ & 0.34 & -0.907  \\
& (0.847) & (0.010) & (0.450) & \\[0.5em]

Number of distinct imaging tests & 1.34  & $0.780^{***}$ & $1.835^{***}$ & 1.23 \\
& (0.563) & (0.009) & (0.320) & \\[0.5em]

72hr return with admission & 0.0123  & $-0.0048^{***}$ & 0.0158 & 0.617  \\
& (0.110) & (0.001) & (0.038) & \\[0.5em]

\midrule
Time FE & --- & Yes & Yes & --- \\
Baseline controls & --- & Yes & Yes & --- \\
Observations & --- & 42,085 & 42,085 & --- \\
\bottomrule
\end{tabular}
\begin{flushleft}
\footnotesize
\textit{Notes:} This table reports OLS and two-stage least squares estimates from the MGH dataset. All specifications include time-fixed effects, baseline controls, and care-area fixed effects. Robust standard errors clustered at the physician level are reported in parentheses. Column 1 reports the mean and standard deviation for non-batched patients (sequenced). Column 4 reports the Z-statistic from a formal test comparing the 2SLS coefficient in Column 3 to the 2SLS coefficient in Column 5 of Table \ref{tab:results_table}.
Significance levels: $^{***}p < 0.001$
\end{flushleft}
\end{threeparttable}
\end{table}

\subsection{Managerial Implications}\label{managerial-implications}

Our findings have important implications for the management of ED
operations. First, while potentially appealing as a workflow efficiency
strategy, batch ordering of imaging tests significantly increases ED LOS
and resource utilization without corresponding improvements in patient
outcomes. The substantial magnitude of these effects---a 65\% increase
in LOS and 88\% increase in imaging tests for discretionarily batched
patients---suggests that ED managers should carefully evaluate policies
around physician test ordering. As our mediation analysis reveals, these
delays arise primarily through increased diagnostic intensity rather
than the batching process itself, indicating that standard practice
(sequential ordering) serves as a natural filter against unnecessary
testing. These findings support Hypotheses 1--3: discretionary batching
increases test volume, lengthens processing times, and raises admission
probability.

The significant variation we observe in batching behavior across
physicians treating similar patients suggests an opportunity for
standardization. ED managers might consider implementing
decision-support systems that encourage sequential ordering,
particularly in conditions where information from initial tests often
eliminates the need for additional imaging. Notably, our heterogeneity
analysis shows that the operational inefficiencies of batching persist
across all capacity levels---effects are statistically indistinguishable
whether the ED is operating normally or under overcapacity (contrary to
Hypothesis 4). This suggests that interventions to reduce unnecessary
batching are warranted across all operational contexts, not only during
periods of normal capacity.

The substantial cost implications of batch ordering---both in terms of
operational efficiency and resource utilization---suggest that EDs could
benefit from more structured approaches to test ordering. While
preserving physician autonomy in clinical decision-making is crucial,
our results indicate that unfettered discretion in test ordering timing
may lead to suboptimal system performance. Simple interventions, such as
providing physicians with feedback on their batching rates relative to
peers or implementing decision-support systems that suggest sequential
ordering pathways, could help reduce unnecessary testing while
maintaining care quality. Our determinants analysis suggests that such
interventions may be particularly valuable early in physician shifts,
when batching is most prevalent.

Finally, our findings have implications beyond individual EDs, as
increased admission rates from batch ordering create spillover effects
throughout the hospital system. Discretionary batching increases
admission probability by 40 percentage points, raising patient volumes
in inpatient units and potentially inducing speed-up behaviors that can
compromise care quality. Hospital administrators should consider these
downstream impacts when developing imaging protocols and resource
allocation strategies. The consistency of our findings across two
institutions with different patient populations and assignment
mechanisms (Mayo Clinic and MGH) suggests these patterns are not
idiosyncratic to a single setting, strengthening the case for broader
policy attention to diagnostic test sequencing in emergency care.

\subsection{Robustness Checks}\label{sec:robustness}

While our research design leverages the random assignment of patients to
physicians to identify causal effects, several limitations warrant
discussion. A primary concern is that physicians' tendency to batch
order tests may correlate with other unobserved practice patterns that
affect our outcomes of interest. Although we found no significant
associations between observable physician characteristics (such as
experience, gender, or training) and the tendency to batch, unobserved
characteristics could influence both the tendency to batch and other
aspects of patient care. For instance, physicians who tend to
batch-order tests might also have different approaches to patient
assessment, documentation practices, or consultation patterns, each of
which can independently affect LOS and disposition decisions. Removing
the potential impact of such unobserved factors might require running a
fully randomized experiment. However, the consistency and magnitude of
our findings across (a) both 2SLS and UJIVE specifications, (b)
specifications with and without precision controls for laboratory
ordering and physician characteristics, and (c) two different hospitals
with different practice settings provide strong evidence behind our
findings. In particular, unobserved physician characteristics would need
to be substantial to invalidate our core finding that batching increases
both test utilization and processing times.

Nevertheless, the validity of our results largely depends on our
identification strategy (Section \ref{sec:identification}). Thus, we
performed several robustness checks (see Appendix D) to further
strengthen our confidence in the validity of our results. To address
potential many-instrument bias in leniency designs, we implement the
Unbiased Jackknife Instrumental Variables Estimator (UJIVE), which uses
leave-one-out predictions to eliminate mechanical correlation between
the instrument and structural error
\citep{kolesar2015manyinvalid, goldsmithpinkham2025leniencydesignsoperatorsmanual}.
The close agreement between 2SLS and UJIVE estimates across all outcomes
(Table \ref{tab:results_table}) increases confidence that
many-instrument bias is not driving our findings. To address potential
exclusion restriction violations, we conduct sensitivity analyses
following \citet{conley2012plausibly} that allow for plausible direct
effects of the instrument on outcomes; our estimates remain significant
across a wide range of assumed violation magnitudes (Appendix Figure
D2). Third, we conduct a placebo test examining complaints in which
imaging is not ordered and find null effects, supporting the idea that
batch tendency captures batching-specific behavior (Appendix Table D3).

We also verify robustness to alternative outcome measures. We examine
treatment time (from physician contact to disposition), which excludes
both waiting room delays and post-disposition boarding; results are
qualitatively unchanged (Appendix Table D4). We also verify robustness
to measuring 72-hour returns regardless of admission status; effects
remain small and statistically indistinguishable from zero.
Additionally, we confirm robustness to nonlinear model
specifications---logit for binary outcomes and Poisson for counts---with
average marginal effects aligning closely with our linear IV estimates
(Appendix Table D5).

To assess generalizability across clinical scenarios, we conduct
heterogeneity analyses across seven chief complaint categories that vary
in complexity and prevalence of batching. We find consistent
patterns---increased imaging without efficiency gains or quality
improvements---across all complaint types (Figure
\ref{fig:heterogeneity}), strengthening confidence that specific
clinical presentations do not drive our findings.

Finally, to address monotonicity, we follow the approaches of
\citet{dobbie2018effects} and \citet{bhuller2020incarceration} and check
whether our instrument shows a positive first-stage coefficient across
key complaint, severity, capacity, and demographic-related subsamples
(Appendix Figure D1). Across all robustness checks and main outcomes, we
find that the magnitudes of our estimates remain stable, and the
implications of our results remain consistent. Our robustness checks
thus provide us with further confidence in the validity of our main
findings.

\section{Conclusion}\label{conclusion}

Although prior literature examines task ordering from the perspective of
centralized protocols, frontline physicians often have discretion over
how and when they order diagnostic tests, especially in high-pressure
environments like EDs. In practice, due to system design or individual
choice, this delegation of decision-making results in physicians
self-managing their test-ordering strategies. Given the limited research
on the operational implications of test ordering, little is known about
how healthcare managers should guide these practices when such
discretion exists. Understanding when and how physicians exercise this
discretion informs decisions about system design, how to encourage
optimal use of discretion, and how to adjust policies to account for
frontline clinicians' behaviors.

We explore this underexamined area by analyzing the drivers and
consequences of batch ordering imaging tests in the ED. Utilizing
detailed operational data from two large U.S. EDs, one with random
patient-physician assignment, we find that physicians are more likely to
batch order imaging tests earlier in their shifts and when the ED is
less crowded. This suggests that time pressure and occupancy levels
significantly influence the decision to batch, consistent with the
notion that physicians may adjust their ordering behavior in response to
their workload and the operational demands of the ED.

Our results indicate that batch ordering imaging tests significantly
increases the number of imaging studies performed per patient encounter,
confirming that batching contributes to higher resource utilization.
However, we do not find evidence that batching reduces patient LOS in
the ED or impacts the likelihood of a 72-hour return with admission. On
the contrary, discretionary batching substantially increases time spent
in the ED and time to disposition, suggesting significant inefficiencies
arise from initiating multiple diagnostic imaging processes
simultaneously. This result aligns with previous research emphasizing
the importance of diagnostic pathways in achieving optimal health
outcomes and operational efficacy \citep[\citet{Masic2008},
\citet{Singh2015}]{Carpenter2015}. This is consistent with the
information gain advantage of sequential test ordering, where the
results of one test may eliminate the need for another. Our findings
suggest that discretion allows physicians to tailor their test ordering
to specific situations. However, it may also lead to resource-intensive
practices that do not benefit patients or the ED system.

Moreover, excess testing in EDs is not a benign phenomenon. It is
associated with increased risks, including patient exposure to
unnecessary radiation and the resultant psychological and physical
burden from incidental findings \citep{Mus2011}. Moreover, the economic
implications are substantial, with the overuse of diagnostic tests
contributing significantly to the escalating costs of healthcare
\citep{Atkinson2023}. As such, our results suggest the need to examine
the practice of batching across different clinical conditions and in
other clinical settings beyond the ED while also considering its
consequences across a variety of metrics that affect hospitals' publicly
reported outcomes \citep[\citet{saghafian2019role}]{Saghafian2019}.

Our study underscores the importance of developing evidence-based
guidelines to inform physicians' test ordering strategies. By
understanding how batching impacts patient outcomes and ED operations,
healthcare managers can design interventions to optimize test ordering
practices. This may include providing decision support tools, adjusting
policies to encourage sequential ordering when appropriate, or offering
feedback to physicians on their ordering patterns and associated
outcomes. By aligning physician test-ordering strategies more closely
with patient needs \citep{Atkinson2023}, EDs can enhance patient
satisfaction and outcomes while improving operational efficiency.

\clearpage

% Appendix here
% Options are (1) APPENDIX (with or without general title) or
%             (2) APPENDICES (if it has more than one unrelated sections)
% Outcomment the appropriate case if necessary
%
% \begin{APPENDIX}{<Title of the Appendix>}
% \end{APPENDIX}
%
%   or
%
% \begin{APPENDICES}
% \section{<Title of Section A>}
% \section{<Title of Section B>}
% etc
% \end{APPENDICES}


% Acknowledgments here
\ACKNOWLEDGMENT{}

\bibliographystyle{informs2014}
\bibliography{references.bib}



\end{document}
